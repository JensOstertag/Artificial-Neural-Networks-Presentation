\documentclass[
	aspectratio=169,
	8pt
]{beamer}

\usepackage[ngerman]{babel}
\usepackage{bibgerm}
\usepackage{csquotes}
\usepackage{hyperref}
\usepackage{wrapfig}
\usepackage{xcolor}
\usepackage{graphicx}
\usepackage{amssymb}
\usepackage{tikz}
\usetikzlibrary{shapes.geometric}

\title[Künstliche neuronale Netze]{Neuronale Netze: Deep Convolutional Neural Networks}
\author{Jens Ostertag}
\date{6. Juli 2022}

\definecolor{presentationRed}{RGB}{164, 6, 11}

\usepackage{beamerthemeuulm}
\usetheme{uulm}
\setfaculty{infIngPsy}

\begin{document}
\maketitle[apr21-o25a][250]

\section{Computer lernen lassen}
\subsection{Allgemeines}
\begin{frame}[t]{\insertsubsection}
Wo werden neuronale Netze angewendet?
\begin{itemize}
\item Straßenverkehr
\item Medizinischer Bereich
\item Industrie
\item Soziale Netzwerke
\end{itemize}
\end{frame}

\subsection{Aufbau neuronaler Netze}
\begin{frame}[t]{\insertsubsection}
\begin{itemize}
\item Sehr stark an natürliche neuronale Netze angelehnt
\begin{itemize}
\item Neuronen mit Synapsen verbunden
\item Synapsengewicht regelt Stromfluss zwischen zwei Neuronen
\end{itemize}
\end{itemize}

\vspace*{\fill}
\begin{center}
\scalebox{.75}{
\begin{tikzpicture}[thick, main/.style={draw, circle, inner sep=0pt, minimum width=25pt}]
\node (a1) at (1, 2) {\dots};
\node (a2) at (0, 0) {\dots};
\node (a3) at (1, -2) {\dots};

\node (b1) at (5, 2) {\dots};
\node[main, fill=blue!25] (b2) at (3, 0) {$n_{0, 0}$};
\node (b3) at (5, -2) {\dots};

\node (w1) at (5, 0) {$w_{0, 0, 0}$};

\node[main, fill=blue!25] (c2) at (7, 0) {$n_{1, 0}$};
\node (d2) at (9, 0) {\dots};

\draw[-] (a1) -- (b2);
\draw[-] (a2) -- (b2);
\draw[-] (a3) -- (b2);

\draw[-] (b1) -- (c2);
\draw[-] (b2) -- (w1) -- (c2);
\draw[-] (b3) -- (c2);

\draw[-] (c2) -- (d2);
\end{tikzpicture}
}
\end{center}
\vspace*{\fill}
\end{frame}

\begin{frame}[t]{\insertsubsection}
\begin{itemize}
\item Ausgaben von Neuronen und Synapsengewichte werden als Zahl betrachtet $\rightarrow$ Berechenbarkeit
\item Einteilung der Neuronen in unterschiedliche Schichten
\begin{itemize}
\item Input Layer
\item Hidden Layer
\item Output Layer
\end{itemize}
\end{itemize}

\vspace*{\fill}
\begin{center}
\scalebox{.5}{
\begin{tikzpicture}[thick, main/.style={draw, circle, inner sep=0pt, minimum width=25pt}]
\node[main, fill=red!25] (i1) at (0, 0.75) {};
\node[main, fill=red!25] (i2) at (0, 2.25) {};  
\node[main, fill=red!25] (i3) at (0, 3.75) {};
\node[main, fill=red!25] (i4) at (0, 5.25) {};

\node[main, fill=blue!25] (h11) at (3, 0) {};  
\node[main, fill=blue!25] (h12) at (3, 1.5) {};
\node[main, fill=blue!25] (h13) at (3, 3) {};
\node[main, fill=blue!25] (h14) at (3, 4.5) {};
\node[main, fill=blue!25] (h15) at (3, 6) {};

\node[main, fill=blue!25] (h21) at (6, 0) {};  
\node[main, fill=blue!25] (h22) at (6, 1.5) {};
\node[main, fill=blue!25] (h23) at (6, 3) {};
\node[main, fill=blue!25] (h24) at (6, 4.5) {};
\node[main, fill=blue!25] (h25) at (6, 6) {};

\node[main, fill=green!35] (o1) at (9, 1.5) {};
\node[main, fill=green!35] (o2) at (9, 3) {};
\node[main, fill=green!35] (o3) at (9, 4.5) {};

\draw[-] (i1) -- (h11);\draw[-] (i1) -- (h12);\draw[-] (i1) -- (h13);\draw[-] (i1) -- (h14);\draw[-] (i1) -- (h15);

\draw[-] (i2) -- (h11);\draw[-] (i2) -- (h12);\draw[-] (i2) -- (h13);\draw[-] (i2) -- (h14);\draw[-] (i2) -- (h15);

\draw[-] (i3) -- (h11);\draw[-] (i3) -- (h12);\draw[-] (i3) -- (h13);\draw[-] (i3) -- (h14);\draw[-] (i3) -- (h15);

\draw[-] (i4) -- (h11);\draw[-] (i4) -- (h12);\draw[-] (i4) -- (h13);\draw[-] (i4) -- (h14);\draw[-] (i4) -- (h15);

\draw[-] (h11) -- (h21);\draw[-] (h11) -- (h22);\draw[-] (h11) -- (h23);\draw[-] (h11) -- (h24);\draw[-] (h11) -- (h25);

\draw[-] (h12) -- (h21);\draw[-] (h12) -- (h22);\draw[-] (h12) -- (h23);\draw[-] (h12) -- (h24);\draw[-] (h12) -- (h25);

\draw[-] (h13) -- (h21);\draw[-] (h13) -- (h22);\draw[-] (h13) -- (h23);\draw[-] (h13) -- (h24);\draw[-] (h13) -- (h25);

\draw[-] (h14) -- (h21);\draw[-] (h14) -- (h22);\draw[-] (h14) -- (h23);\draw[-] (h14) -- (h24);\draw[-] (h14) -- (h25);

\draw[-] (h15) -- (h21);\draw[-] (h15) -- (h22);\draw[-] (h15) -- (h23);\draw[-] (h15) -- (h24);\draw[-] (h15) -- (h25);

\draw[-] (o1) -- (h21);\draw[-] (o1) -- (h22);\draw[-] (o1) -- (h23);\draw[-] (o1) -- (h24);\draw[-] (o1) -- (h25);

\draw[-] (o2) -- (h21);\draw[-] (o2) -- (h22);\draw[-] (o2) -- (h23);\draw[-] (o2) -- (h24);\draw[-] (o2) -- (h25);

\draw[-] (o3) -- (h21);\draw[-] (o3) -- (h22);\draw[-] (o3) -- (h23);\draw[-] (o3) -- (h24);\draw[-] (o3) -- (h25);
\end{tikzpicture}
}
\end{center}
\vspace*{\fill}
\end{frame}

\begin{frame}[t]{\insertsubsection}
\vspace*{\fill}
\myexample{Berechnung der Ausgabe}{
\[
o_{i, j} = \varphi \left( \sum\limits_{k=0}^{|n_{i-1}| - 1} o_{i-1, k} * w_{i-1, k, j} \right)
\]
\begin{itemize}
\item $n_{i, j}$: Neuron in der Schicht $i$ an der Stelle $j$
\item $o_{i, j}$: Ausgabe des Neurons $n_{i, j}$
\item $\varphi$: Differenzierbare Aktivierungsfunktion
\item $|n_i|$: Anzahl der Neuronen in der Schicht $i$
\item $w_{i, k, j}$: Synapsengewicht zwischen den Neuronen $n_{i, k}$ und $n_{i+1, j}$
\end{itemize}
}
\vspace*{\fill}
\end{frame}

\subsection{Arten des maschinellen Lernens}
\begin{frame}[t]{\insertsubsection}
\begin{itemize}
\item Überwachtes Lernen
\begin{itemize}
\item Lernen mit einem vorgegebenen Datensatz
\item Erkennen von Eigenschaften der Daten
\item z.B. zur Klassifikation von Daten
\end{itemize}
\item Unüberwachtes Lernen
\begin{itemize}
\item Erkennen von Ähnlichkeiten in einer Datenmenge
\item z.B. zur Gruppierung von Daten
\end{itemize}
\item Bestärktes Lernen
\begin{itemize}
\item Lernen mit einem Belohnungssystem
\item z.B. zur Durchführung von Aufgaben
\end{itemize}
\end{itemize}
\end{frame}

\subsection{Umgang mit neuronalen Netzen}
\begin{frame}[t]{\insertsubsection}
Hier am Beispiel des überwachten Lernens
\begin{itemize}
\item Vorbereiten eines Trainingsdatensatzes
\begin{itemize}
\item Beispieldaten werden händisch klassifiziert
\item Evtl. einheitliche Bearbeitung von Daten
\item Erstellen eines Testdatensatzes zur Auswertung
\end{itemize}
\item Trainieren
\begin{itemize}
\item Trainieren mit unterschiedlichen Algorithmen
\end{itemize}
\item Auswertung und Anwendung
\begin{itemize}
\item Auswertung mithilfe des Testdatensatzes
\item Exportieren der Netzstruktur und der Gewichte
\item Importieren in Anwendung
\end{itemize}
\end{itemize}
\end{frame}

\section{Deep Convolutional Neural Networks}
\subsection{Abwandlung neuronaler Netze}
\begin{frame}[t]{\insertsubsection}
Warum werden Convolutional Neural Networks benötigt?
\begin{itemize}
\item Zu lange Trainingsdauer
\item Art und Weise der Erkennung eines Objekts
\begin{itemize}
\item Zwei Pixel-Paare sind gleich relevant, unabhängig von Entfernung
\end{itemize}
\item[\color{presentationRed}$\Rightarrow$] Risiko des \textbf{Overfittings}
\end{itemize}
Vorteile:
\begin{itemize}
\item Viel bessere Genauigkeit
\begin{itemize}
\item Lernvorgänge noch ähnlicher zu denen des Menschen
\end{itemize}
\item Schnelleres Training
\begin{itemize}
\item Weniger anzupassende Gewichte
\end{itemize}
\end{itemize}
\end{frame}

\subsection{Convolution / Cross Correlation im eindimensionalen Raum}
\begin{frame}[t]{\insertsubsection}
\myexample{Berechnung der Cross Correlation}{
\[
y_i = \sum\limits_{k=0}^{|w|-1} x_{i+k}*w_k
\]
\begin{itemize}
\item \small$y_i$: Ausgabe an der Stelle $i$
\item \small$x_i$: Eingabe an der Stelle $i$
\item \small$w_i$: Wert des Kernels an der Stelle $i$
\end{itemize}
}

\vspace*{\fill}
\begin{center}
\scalebox{.6}{
\begin{tikzpicture}[thick, main/.style={draw, rectangle, inner sep=0pt, minimum width=1cm, minimum height=1cm}]
\node at (-1.5, .25) {$x$};
\node[text=blue!50] at (-1, -.25) {$w$};
\node at (-.5, -2.5) {$y$};

\node[main] at (0, 0) {};
\node[main] at (1, 0) {};
\node[main] at (2, 0) {};
\node[main] at (3, 0) {};
\node[main] at (4, 0) {};

\node[main, draw=blue!35, line width=.75mm] at (.25, -.25) {};
\node[main, draw=blue!35, line width=.75mm] at (1.25, -.25) {};
\node[main, draw=blue!35, line width=.75mm] at (2.25, -.25) {};

\node[main, fill=blue!25] at (1, -2.5) {};
\node[main] at (2, -2.5) {};
\node[main] at (3, -2.5) {};

\draw[dashed, draw=blue!25, line width=.5mm] (-.25, -.75) -- (.5, -3);
\draw[dashed, draw=blue!25, line width=.5mm] (-.25, .25) -- (.5, -2);
\draw[dashed, draw=blue!25, line width=.5mm] (2.75, -.75) -- (1.5, -3);
\draw[dashed, draw=blue!25, line width=.5mm] (2.75, .25) -- (1.5, -2);
\end{tikzpicture}}
$\qquad$
\scalebox{.6}{\begin{tikzpicture}[thick, main/.style={draw, rectangle, inner sep=0pt, minimum width=1cm, minimum height=1cm}]
\node[main] at (0, 0) {};
\node[main] at (1, 0) {};
\node[main] at (2, 0) {};
\node[main] at (3, 0) {};
\node[main] at (4, 0) {};

\node[main, draw=blue!35, line width=.75mm] at (1.25, -.25) {};
\node[main, draw=blue!35, line width=.75mm] at (2.25, -.25) {};
\node[main, draw=blue!35, line width=.75mm] at (3.25, -.25) {};

\node[main] at (1, -2.5) {};
\node[main, fill=blue!25] at (2, -2.5) {};
\node[main] at (3, -2.5) {};

\draw[dashed, draw=blue!25, line width=.5mm] (.75, -.75) -- (1.5, -3);
\draw[dashed, draw=blue!25, line width=.5mm] (.75, .25) -- (1.5, -2);
\draw[dashed, draw=blue!25, line width=.5mm] (3.75, -.75) -- (2.5, -3);
\draw[dashed, draw=blue!25, line width=.5mm] (3.75, .25) -- (2.5, -2);
\end{tikzpicture}}
\end{center}
\vspace*{\fill}
\end{frame}

\subsection{Padding}
\begin{frame}[t]{\insertsubsection}
Problem: Ausgabe wird kleiner als Eingabe
\begin{itemize}
\item[\color{presentationRed}$\Rightarrow$] Detailverlust
\end{itemize}
Lösung: Eingabe mit Nullen erweitern

\vspace*{\fill}
\begin{center}
\scalebox{.6}{
\begin{tikzpicture}[thick, main/.style={draw, rectangle, inner sep=0pt, minimum width=1cm, minimum height=1cm}]
\node at (-2.5, .25) {$x^p$};
\node[text=blue!50] at (-2, -.25) {$w$};
\node at (-1.5, -2.5) {$y$};

\node[main, dotted] at (-1, 0) {$0$};
\node[main] at (0, 0) {};
\node[main] at (1, 0) {};
\node[main] at (2, 0) {};
\node[main] at (3, 0) {};
\node[main] at (4, 0) {};
\node[main, dotted] at (5, 0) {$0$};

\node[main, draw=blue!35, line width=.75mm] at (-.75, -.25) {};
\node[main, draw=blue!35, line width=.75mm] at (.25, -.25) {};
\node[main, draw=blue!35, line width=.75mm] at (1.25, -.25) {};

\node[main, fill=blue!25] at (0, -2.5) {};
\node[main] at (1, -2.5) {};
\node[main] at (2, -2.5) {};
\node[main] at (3, -2.5) {};
\node[main] at (4, -2.5) {};

\draw[dashed, draw=blue!25, line width=.5mm] (-1.25, -.75) -- (-.5, -3);
\draw[dashed, draw=blue!25, line width=.5mm] (-1.25, .25) -- (-.5, -2);
\draw[dashed, draw=blue!25, line width=.5mm] (1.75, -.75) -- (.5, -3);
\draw[dashed, draw=blue!25, line width=.5mm] (1.75, .25) -- (.5, -2);
\end{tikzpicture}}
\end{center}
\vspace*{\fill}
\end{frame}

\subsection{Convolution / Cross Correlation im zweidimensionalen Raum}
\begin{frame}[t]{\insertsubsection}
\vspace*{\fill}
\begin{center}
\scalebox{.6}{
\begin{tikzpicture}
\node at (2.5, 1.5) {$X^p$};
\node[text=blue!50] at (.75, 2) {$W$};
\node at (9.5, 1) {$Y$};
\draw[-] (0, 0) -- (5, -1) -- (5, -6) -- (0, -5) -- (0, 0);
\draw[-] (0, -1) -- (5, -2);
\draw[-] (0, -2) -- (5, -3);
\draw[-] (0, -3) -- (5, -4);
\draw[-] (0, -4) -- (5, -5);
\draw[-] (1, -.2) -- (1, -5.2);
\draw[-] (2, -.4) -- (2, -5.4);
\draw[-] (3, -.6) -- (3, -5.6);
\draw[-] (4, -.8) -- (4, -5.8);
\draw[dotted] (-1, 1.2) -- (6, -.2) -- (6, -7.2) -- (-1, -5.8) -- (-1, 1.2);
\draw[dotted] (0, 1) -- (0, -6);
\draw[dotted] (1, .8) -- (1, -6.2);
\draw[dotted] (2, .6) -- (2, -6.4);
\draw[dotted] (3, .4) -- (3, -6.6);
\draw[dotted] (4, .2) -- (4, -6.8);
\draw[dotted] (5, 0) -- (5, -7);
\draw[dotted] (-1, .2) -- (6, -1.2);
\draw[dotted] (-1, -.8) -- (6, -2.2);
\draw[dotted] (-1, -1.8) -- (6, -3.2);
\draw[dotted] (-1, -2.8) -- (6, -4.2);
\draw[dotted] (-1, -3.8) -- (6, -5.2);
\draw[dotted] (-1, -4.8) -- (6, -6.2);
\node at (-.5, .6) {$0$};
\node at (.5, .4) {$0$};
\node at (1.5, .2) {$0$};
\node at (2.5, 0) {$0$};
\node at (3.5, -.2) {$0$};
\node at (4.5, -.4) {$0$};
\node at (5.5, -.6) {$0$};
\node at (5.5, -1.6) {$0$};
\node at (5.5, -2.6) {$0$};
\node at (5.5, -3.6) {$0$};
\node at (5.5, -4.6) {$0$};
\node at (5.5, -5.6) {$0$};
\node at (5.5, -6.6) {$0$};
\node at (4.5, -6.4) {$0$};
\node at (3.5, -6.2) {$0$};
\node at (2.5, -6) {$0$};
\node at (1.5, -5.8) {$0$};
\node at (.5, -5.6) {$0$};
\node at (-.5, -5.4) {$0$};
\node at (-.5, -4.4) {$0$};
\node at (-.5, -3.4) {$0$};
\node at (-.5, -2.4) {$0$};
\node at (-.5, -1.4) {$0$};
\node at (-.5, -.4) {$0$};
\draw[-, draw=blue!35, line width=.75mm] (-.75, .95) -- (2.25, .35) -- (2.25, -2.65) -- (-.75, -2.05) -- (-.75, .95);
\draw[-, draw=blue!35, line width=.75mm] (.25, .75) -- (.25, -2.25);
\draw[-, draw=blue!35, line width=.75mm] (1.25, .55) -- (1.25, -2.45);
\draw[-, draw=blue!35, line width=.75mm] (-.75, -.05) -- (2.25, -.65);
\draw[-, draw=blue!35, line width=.75mm] (-.75, -1.05) -- (2.25, -1.65);
\fill[fill=blue!25] (7, 0) -- (8, -.2) -- (8, -1.2) -- (7, -1);
\draw[-] (7, 0) -- (12, -1) -- (12, -6) -- (7, -5) -- (7, 0);
\draw[-] (7, -1) -- (12, -2);
\draw[-] (7, -2) -- (12, -3);
\draw[-] (7, -3) -- (12, -4);
\draw[-] (7, -4) -- (12, -5);
\draw[-] (8, -.2) -- (8, -5.2);
\draw[-] (9, -.4) -- (9, -5.4);
\draw[-] (10, -.6) -- (10, -5.6);
\draw[-] (11, -.8) -- (11, -5.8);
\draw[dashed, draw=blue!25, line width=.5mm] (-.75, .95) -- (7, 0);
\draw[dashed, draw=blue!25, line width=.5mm] (2.25, .35) -- (8, -.2);
\draw[dashed, draw=blue!25, line width=.5mm] (2.25, -2.65) -- (8, -1.2);
\draw[dashed, draw=blue!25, line width=.5mm] (-.75, -2.05) -- (7, -1);
\end{tikzpicture}
}
\end{center}
\vspace*{\fill}
\end{frame}

\subsection{Convolutional Layer}
\begin{frame}[t]{\insertsubsection}
\begin{itemize}
\item Mehrere Eingabechannel (zum Beispiel unterschiedliche Farbkanäle)
\item Mehrere Ausgabechannel
\item Jeder Eingabechannel hat Einfluss auf jeden Ausgabechannel
\begin{itemize}
\item[\color{presentationRed}$\Rightarrow$] Mehrere Kernel pro Convolutional Layer
\end{itemize}
\item Ausgabe eines Convolutional Layer: Feature Map
\begin{itemize}
\item Allgemeine Eigenschaften in höheren Schichten
\item Komplexere Eigenschaften in tieferen Schichten
\end{itemize}
\end{itemize}
\end{frame}

\subsection{Subsampling}
\begin{frame}[t]{\insertsubsection}
\begin{itemize}
\item Verringerung der Eigenschaften durch Pooling-Operationen
\item Schnellerer Trainingsvorgang, nur geringer Leistungsverlust
\end{itemize}
\vspace*{\fill}
\vspace*{\fill}
\begin{center}
\scalebox{.9}{
\begin{tikzpicture}[thick, main/.style={draw, rectangle, inner sep=0pt, minimum width=.5cm, minimum height=.5cm}]
\node[main, fill=blue!25] at (0, 0) {3};
\node[main, fill=blue!25] at (.5, 0) {1};
\node[main, fill=orange!35] at (1, 0) {5};
\node[main, fill=orange!35] at (1.5, 0) {1};

\node[main, fill=blue!25] at (0, .5) {2};
\node[main, fill=blue!25] at (.5, .5) {6};
\node[main, fill=orange!35] at (1, .5) {2};
\node[main, fill=orange!35] at (1.5, .5) {4};

\node[main, fill=red!25] at (0, 1) {9};
\node[main, fill=red!25] at (.5, 1) {3};
\node[main, fill=green!35] at (1, 1) {8};
\node[main, fill=green!35] at (1.5, 1) {1};

\node[main, fill=red!25] at (0, 1.5) {7};
\node[main, fill=red!25] at (.5, 1.5) {1};
\node[main, fill=green!35] at (1, 1.5) {2};
\node[main, fill=green!35] at (1.5, 1.5) {5};

\draw[->] (2.5, .75) -- (3.5, .75);

\node at (4.75, 2.5) {Max-Pooling};
\node[main, fill=red!25] at (4.5, 1.75) {9};
\node[main, fill=green!25] at (5, 1.75) {8};
\node[main, fill=blue!25] at (4.5, 1.25) {6};
\node[main, fill=orange!25] at (5, 1.25) {5};

\node at (4.75, -1) {Mean-Pooling};
\node[main, fill=red!25] at (4.5, .25) {5};
\node[main, fill=green!25] at (5, .25) {4};
\node[main, fill=blue!25] at (4.5, -.25) {3};
\node[main, fill=orange!25] at (5, -.25) {3};
\end{tikzpicture}
}
\end{center}
\vspace*{\fill}

\begin{itemize}
\item Keine anzupassenden Gewichte
\end{itemize}
\end{frame}

\subsection{Aufbau eines Convolutional Neural Networks}
\begin{frame}[t]{\insertsubsection}
\vspace*{\fill}
\begin{center}
\begin{tikzpicture}[thick, main/.style={draw, circle, inner sep=0pt, minimum width=.25cm}]
\def\lgt{1.5};
\def\xo{0};
\def\yo{0};
\def\dst{.2};
\draw[-] (0+\xo, 0+\yo) -- (0+\xo, \lgt+\yo) -- (\lgt+\xo, \lgt+\yo) -- (\lgt+\xo, 0+\yo) -- (0+\xo, 0+\yo);
\foreach \n in {0, ..., 1} {
\draw[-] (-\dst*\n+\xo, \dst*\n+\dst+\yo) -- (-\dst*\n-\dst+\xo, \dst*\n+\dst+\yo) -- (-\dst*\n-\dst+\xo, \dst*\n+\dst+\lgt+\yo) -- (-\dst*\n-\dst+\lgt+\xo, \dst*\n+\dst+\lgt+\yo) -- (-\dst*\n-\dst+\lgt+\xo, \dst*\n+\lgt+\yo);
}

\def\lgt{1.5};
\def\xo{2.5};
\def\yo{-.15};
\def\dst{.1};
\draw[-] (0+\xo, 0+\yo) -- (0+\xo, \lgt+\yo) -- (\lgt+\xo, \lgt+\yo) -- (\lgt+\xo, 0+\yo) -- (0+\xo, 0+\yo);
\foreach \n in {0, ..., 6} {
\draw[-] (-\dst*\n+\xo, \dst*\n+\dst+\yo) -- (-\dst*\n-\dst+\xo, \dst*\n+\dst+\yo) -- (-\dst*\n-\dst+\xo, \dst*\n+\dst+\lgt+\yo) -- (-\dst*\n-\dst+\lgt+\xo, \dst*\n+\dst+\lgt+\yo) -- (-\dst*\n-\dst+\lgt+\xo, \dst*\n+\lgt+\yo);
}

\def\lgt{1};
\def\xo{5};
\def\yo{.15};
\def\dst{.1};
\draw[-] (0+\xo, 0+\yo) -- (0+\xo, \lgt+\yo) -- (\lgt+\xo, \lgt+\yo) -- (\lgt+\xo, 0+\yo) -- (0+\xo, 0+\yo);
\foreach \n in {0, ..., 6} {
\draw[-] (-\dst*\n+\xo, \dst*\n+\dst+\yo) -- (-\dst*\n-\dst+\xo, \dst*\n+\dst+\yo) -- (-\dst*\n-\dst+\xo, \dst*\n+\dst+\lgt+\yo) -- (-\dst*\n-\dst+\lgt+\xo, \dst*\n+\dst+\lgt+\yo) -- (-\dst*\n-\dst+\lgt+\xo, \dst*\n+\lgt+\yo);
}

\def\lgt{1};
\def\xo{7.5};
\def\yo{-.05};
\def\dst{.075};
\draw[-] (0+\xo, 0+\yo) -- (0+\xo, \lgt+\yo) -- (\lgt+\xo, \lgt+\yo) -- (\lgt+\xo, 0+\yo) -- (0+\xo, 0+\yo);
\foreach \n in {0, ..., 14} {
\draw[-] (-\dst*\n+\xo, \dst*\n+\dst+\yo) -- (-\dst*\n-\dst+\xo, \dst*\n+\dst+\yo) -- (-\dst*\n-\dst+\xo, \dst*\n+\dst+\lgt+\yo) -- (-\dst*\n-\dst+\lgt+\xo, \dst*\n+\dst+\lgt+\yo) -- (-\dst*\n-\dst+\lgt+\xo, \dst*\n+\lgt+\yo);
}

\def\lgt{.65};
\def\xo{9.5};
\def\yo{.3};
\def\dst{.05};
\draw[-] (0+\xo, 0+\yo) -- (0+\xo, \lgt+\yo) -- (\lgt+\xo, \lgt+\yo) -- (\lgt+\xo, 0+\yo) -- (0+\xo, 0+\yo);
\foreach \n in {0, ..., 14} {
\draw[-] (-\dst*\n+\xo, \dst*\n+\dst+\yo) -- (-\dst*\n-\dst+\xo, \dst*\n+\dst+\yo) -- (-\dst*\n-\dst+\xo, \dst*\n+\dst+\lgt+\yo) -- (-\dst*\n-\dst+\lgt+\xo, \dst*\n+\dst+\lgt+\yo) -- (-\dst*\n-\dst+\lgt+\xo, \dst*\n+\lgt+\yo);
}

\node[main] (n00) at (11, 0) {};
\node at (11, .7) {$\vdots$};
\node[main] (n01) at (11, 1.2) {};
\node[main] (n02) at (11, 1.6) {};
\node[main] (n03) at (11, 2) {};

\node[main] (n10) at (12, 0) {};
\node at (12, .7) {$\vdots$};
\node[main] (n11) at (12, 1.2) {};
\node[main] (n12) at (12, 1.6) {};
\node[main] (n13) at (12, 2) {};

\node[main] (n20) at (13, .2) {};
\node at (13, .9) {$\vdots$};
\node[main] (n21) at (13, 1.4) {};
\node[main] (n22) at (13, 1.8) {};

\draw[-] (n00) -- (n10);
\draw[-] (n00) -- (n11);
\draw[-] (n00) -- (n12);
\draw[-] (n00) -- (n13);
\draw[-] (n01) -- (n10);
\draw[-] (n01) -- (n11);
\draw[-] (n01) -- (n12);
\draw[-] (n01) -- (n13);
\draw[-] (n02) -- (n10);
\draw[-] (n02) -- (n11);
\draw[-] (n02) -- (n12);
\draw[-] (n02) -- (n13);
\draw[-] (n03) -- (n10);
\draw[-] (n03) -- (n11);
\draw[-] (n03) -- (n12);
\draw[-] (n03) -- (n13);
\draw[-] (n10) -- (n20);
\draw[-] (n10) -- (n21);
\draw[-] (n10) -- (n22);
\draw[-] (n11) -- (n20);
\draw[-] (n11) -- (n21);
\draw[-] (n11) -- (n22);
\draw[-] (n12) -- (n20);
\draw[-] (n12) -- (n21);
\draw[-] (n12) -- (n22);
\draw[-] (n13) -- (n20);
\draw[-] (n13) -- (n21);
\draw[-] (n13) -- (n22);

\path[->, thick] (0.25, 2.1) edge[bend left] node[above, align=center] {\textbf{Convolution}\\Kernel-Größe:\\$8 \times 3 \times x \times y$} (2.5, 2.2);
\path[->, thick] (3.5, -.3) edge[bend right] node[below, align=center] {\textbf{Subsampling}\\Pooling-Größe\\$2 \times 2$} (5.5, 0);
\path[->, thick] (4.8, 2) edge[bend left] node[above, align=center] {\textbf{Convolution}\\Kernel-Größe:\\$16 \times 8 \times x' \times y'$} (6.7, 2.2);
\path[->, thick] (8, -.2) edge[bend right] node[below, align=center] {\textbf{Subsampling}\\Pooling-Größe\\$2 \times 2$} (9.8, .1);
\path[->, thick] (9.1, 1.9) edge[bend left] node[above, align=center] {\textbf{Flattening}} (10.8, 2.2);
\end{tikzpicture}
\end{center}
\vspace*{\fill}
\end{frame}

\section{Praxisbeispiel}
\subsection{Ziffernerkennung}
\begin{frame}[t]{\insertsubsection}
\vspace*{\fill}
\myexample{MNIST-Datensatz}{
\begin{itemize}
\item Handgeschriebene Ziffern von 0 bis 9
\item Schwarz-Weiß-Bilder der Größe $28 \times 28$
\end{itemize}
}
\vspace*{\fill}
\myexample{TensorFlow}{
\begin{itemize}
\item ML-Framework von Google
\item Viele Architekturen neuronaler Netze
\begin{itemize}
\item Allgemeine Datenverarbeitung
\item Bilderkennung
\item Texterkennung
\item Audioerkennung
\item Verstärktes Lernen
\end{itemize}
\item Großer Komfort
\begin{itemize}
\item Einfaches, aber genaues Training
\item GPU-Berechnungen
\end{itemize}
\end{itemize}
}
\vspace*{\fill}
\end{frame}
\end{document}
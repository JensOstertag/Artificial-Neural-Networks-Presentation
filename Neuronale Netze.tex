\documentclass{beamer}

\usepackage[ngerman]{babel}
\usepackage{bibgerm}
\usepackage{csquotes}
\usepackage{hyperref}
\usepackage{wrapfig}
\usepackage{xcolor}
\usepackage{graphicx}
\usepackage{amssymb}
\usepackage{tikz}
\usetikzlibrary{shapes.geometric}

\title[Künstliche neuronale Netze]{\textbf{Künstliche neuronale Netze}\\[5pt]\small Deep Convolutional Neural Networks}
\author{Jens Ostertag}
\date{6. Juli 2022}

\definecolor{presentationRed}{RGB}{164, 6, 11}

\usetheme{CambridgeUS}
\usecolortheme{beaver}
\setbeamertemplate{itemize items}{\color{presentationRed}$\cdot$}

\begin{document}
\begin{frame}
\titlepage
\end{frame}

\begin{frame}[t]{Allgemeines}
Wo werden neuronale Netze angewendet?
\begin{itemize}
\item Straßenverkehr
\item Medizinischer Bereich
\item Industrie
\item Soziale Netzwerke
\end{itemize}
\end{frame}

\begin{frame}[t]{Aufbau neuronaler Netze (1/3)}
\begin{itemize}
\item Sehr stark an natürliche neuronale Netze angelehnt
\begin{itemize}
\item Neuronen mit Synapsen verbunden
\item Synapsengewicht regelt Stromfluss zwischen zwei Neuronen
\end{itemize}
\end{itemize}

\begin{center}
\scalebox{.75}{
\begin{tikzpicture}[thick, main/.style={draw, circle, inner sep=0pt, minimum width=25pt}]
\node (a1) at (1, 2) {\dots};
\node (a2) at (0, 0) {\dots};
\node (a3) at (1, -2) {\dots};

\node (b1) at (5, 2) {\dots};
\node[main, fill=blue!25] (b2) at (3, 0) {$n_{0, 0}$};
\node (b3) at (5, -2) {\dots};

\node (w1) at (5, 0) {$w_{0, 0, 0}$};

\node[main, fill=blue!25] (c2) at (7, 0) {$n_{1, 0}$};
\node (d2) at (9, 0) {\dots};

\draw[dotted] (a1) -- (b2);
\draw[dotted] (a2) -- (b2);
\draw[dotted] (a3) -- (b2);

\draw[dotted] (b1) -- (c2);
\draw[-] (b2) -- (w1) -- (c2);
\draw[dotted] (b3) -- (c2);

\draw[dotted] (c2) -- (d2);
\end{tikzpicture}
}
\end{center}
\end{frame}
\begin{frame}[t]{Aufbau neuronaler Netze (2/3)}
\begin{itemize}
\item Ausgaben von Neuronen und Synapsengewichte werden als Zahl betrachtet $\rightarrow$ Berechenbarkeit
\item Einteilung der Neuronen in unterschiedliche Schichten
\begin{itemize}
\item Input Layer
\item Hidden Layer
\item Output Layer
\end{itemize}
\end{itemize}

\begin{center}
\scalebox{.5}{
\begin{tikzpicture}[thick, main/.style={draw, circle, inner sep=0pt, minimum width=25pt}]
\node[main, fill=red!25] (i1) at (0, 0.75) {};
\node[main, fill=red!25] (i2) at (0, 2.25) {};  
\node[main, fill=red!25] (i3) at (0, 3.75) {};
\node[main, fill=red!25] (i4) at (0, 5.25) {};

\node[main, fill=blue!25] (h11) at (3, 0) {};  
\node[main, fill=blue!25] (h12) at (3, 1.5) {};
\node[main, fill=blue!25] (h13) at (3, 3) {};
\node[main, fill=blue!25] (h14) at (3, 4.5) {};
\node[main, fill=blue!25] (h15) at (3, 6) {};

\node[main, fill=blue!25] (h21) at (6, 0) {};  
\node[main, fill=blue!25] (h22) at (6, 1.5) {};
\node[main, fill=blue!25] (h23) at (6, 3) {};
\node[main, fill=blue!25] (h24) at (6, 4.5) {};
\node[main, fill=blue!25] (h25) at (6, 6) {};

\node[main, fill=green!35] (o1) at (9, 1.5) {};
\node[main, fill=green!35] (o2) at (9, 3) {};
\node[main, fill=green!35] (o3) at (9, 4.5) {};

\draw[-] (i1) -- (h11);\draw[-] (i1) -- (h12);\draw[-] (i1) -- (h13);\draw[-] (i1) -- (h14);\draw[-] (i1) -- (h15);

\draw[-] (i2) -- (h11);\draw[-] (i2) -- (h12);\draw[-] (i2) -- (h13);\draw[-] (i2) -- (h14);\draw[-] (i2) -- (h15);

\draw[-] (i3) -- (h11);\draw[-] (i3) -- (h12);\draw[-] (i3) -- (h13);\draw[-] (i3) -- (h14);\draw[-] (i3) -- (h15);

\draw[-] (i4) -- (h11);\draw[-] (i4) -- (h12);\draw[-] (i4) -- (h13);\draw[-] (i4) -- (h14);\draw[-] (i4) -- (h15);

\draw[-] (h11) -- (h21);\draw[-] (h11) -- (h22);\draw[-] (h11) -- (h23);\draw[-] (h11) -- (h24);\draw[-] (h11) -- (h25);

\draw[-] (h12) -- (h21);\draw[-] (h12) -- (h22);\draw[-] (h12) -- (h23);\draw[-] (h12) -- (h24);\draw[-] (h12) -- (h25);

\draw[-] (h13) -- (h21);\draw[-] (h13) -- (h22);\draw[-] (h13) -- (h23);\draw[-] (h13) -- (h24);\draw[-] (h13) -- (h25);

\draw[-] (h14) -- (h21);\draw[-] (h14) -- (h22);\draw[-] (h14) -- (h23);\draw[-] (h14) -- (h24);\draw[-] (h14) -- (h25);

\draw[-] (h15) -- (h21);\draw[-] (h15) -- (h22);\draw[-] (h15) -- (h23);\draw[-] (h15) -- (h24);\draw[-] (h15) -- (h25);

\draw[-] (o1) -- (h21);\draw[-] (o1) -- (h22);\draw[-] (o1) -- (h23);\draw[-] (o1) -- (h24);\draw[-] (o1) -- (h25);

\draw[-] (o2) -- (h21);\draw[-] (o2) -- (h22);\draw[-] (o2) -- (h23);\draw[-] (o2) -- (h24);\draw[-] (o2) -- (h25);

\draw[-] (o3) -- (h21);\draw[-] (o3) -- (h22);\draw[-] (o3) -- (h23);\draw[-] (o3) -- (h24);\draw[-] (o3) -- (h25);
\end{tikzpicture}
}
\end{center}
\end{frame}

\begin{frame}[t]{Aufbau neuronaler Netze (3/3)}
\begin{itemize}
\item Berechnung der Ausgabe eines Neurons mit der Formel
\[
o_{i, j} = \varphi \left( \sum\limits_{k=0}^{|n_{i-1}| - 1} o_{i-1, k} * w_{i-1, k, j} \right)
\]
\begin{itemize}
\item $n_{i, j}$: Neuron in der Schicht $i$ an der Stelle $j$
\item $o_{i, j}$: Ausgabe des Neurons $n_{i, j}$
\item $\varphi$: Differenzierbare Aktivierungsfunktion
\item $|n_i|$: Anzahl der Neuronen in der Schicht $i$
\item $w_{i, k, j}$: Synapsengewicht zwischen den Neuronen $n_{i, k}$ und $n_{i+1, j}$
\end{itemize}
\end{itemize}
\end{frame}
\end{document}